\documentclass[11 pt]{article}

\setlength{\oddsidemargin}{-.35in}
%\setlength{\evensidemargin}{-.5in}
\setlength{\textwidth}{7.0in}
\setlength{\topmargin}{-0.95in}
\setlength{\textheight}{10.0in}

\usepackage{latexsym}
\usepackage{amsfonts}
\usepackage{amsmath}
\usepackage{amsthm}

\usepackage[T1]{fontenc}
\usepackage{baskervald}
\usepackage[bigdelims,vvarbb]{newtxmath} 

\usepackage{hyperref} 
\hypersetup{colorlinks=true, linkcolor=blue,  anchorcolor=blue,  
citecolor=blue, filecolor=blue, menucolor=blue, pagecolor=blue,  
urlcolor=blue,pdftitle={MTH 124 Algebra Expectations}}

\pagestyle{empty}

%\renewcommand{\baselinestretch}{1.5}

\begin{document}

\noindent MTH 225 Checkpoint 1 \textbf{SAMPLE} \ \ \  \hfill Talbert \\

\noindent {\bfseries Directions:} 

\begin{itemize}
    \item Do only the Checkpoint problems that you need to take and feel ready to take. If you have already earned Mastery on a Learning Target, do not attempt a problem for that Target! You can skip a Target if you need more time to practice with it, and take it on the next round. 
    \item Do not put any work on this form; do all your work on separate pages. You may either handwrite or type up your work. 
    \item Clearly indicate which Learning Target you are attempting at the beginning of its solution; please also turn in solutions for learning targets in order (for example, do not turn in work for A.2 after work for SF.1). The easiest way to do this is to put each Learning Target on its own solution page and do not put more than one Learning Target on a single page. 
    \item If you are handwriting, submit your work by \textbf{scanning your work} using a scanning app or scanning device; \textbf{do not just take a picture} but scan your work to a clear, legible, black and white PDF file of size less than 100 MB. Work submitted as an image file (JPG, PNG, etc.) will not be graded. 
    \item Submit your work by uploading it as a PDF or Word file to the appropriate assignment area on Blackboard.  
\end{itemize}



\noindent \hrulefill

\begin{description}
\item[Learning Target A.1:] \emph{I can represent an integer in base 2, 8, 10, and 16.}

Perform all of the following conversions. Show all work and explain all reasoning. 

\begin{enumerate}
    \item 723 in decimal; convert to binary and octal. 
    \item 10011101 in binary; convert to decimal and hexadecimal. 
    \item 22B in hexadecimal; convert to binary and octal. 
\end{enumerate}

\vspace{0.2in}
\hline

\item[Learning Target A.2 \textbf{(Core)}:] \emph{I can add, subtract, multiply, and divide two integers written in binary.}

Perform all of the following computations in binary, without changing to base 10. Show all work and explain all reasoning. 

\begin{enumerate}
    \item \texttt{110101 + 011011} 
    \item \texttt{110101 - 011011} 
    \item \texttt{11011 $\times$ 111}
    \item \texttt{1101011 $\div$ 111}
\end{enumerate}

\end{description}


\end{document}
