\documentclass[11pt]{article}

\pagestyle{empty}                       %no page numbers
\thispagestyle{empty}                   %removes first page number
\setlength{\parindent}{0in}               %no paragraph indents

\usepackage{fullpage}
\usepackage[tmargin = 0.5in, bmargin = 1in, hmargin = 1in]{geometry}     %1-inch margins
\geometry{letterpaper}                  
\usepackage{graphicx}
\usepackage{amssymb}

% Default packages
\usepackage{latexsym}
\usepackage{amsfonts}
\usepackage{amsmath}
\usepackage{amsthm}
\usepackage{palatino}
\usepackage{hyperref}
\usepackage{multicol}
\usepackage{fancyhdr}
\usepackage{pseudocode}

\begin{document}
	
	\thispagestyle{empty}
	\renewcommand{\headrulewidth}{1pt}
	\thispagestyle{fancy}
	\lhead{Prof. Talbert}
	\chead{MTH 325}
	\rhead{M.20 v0}
	\lfoot{}
	\cfoot{}
	\rfoot{}	
	
	\vspace*{0in}

\noindent 
\textbf{M.20: Determine whether a graph (directed or undirected) has an Euler path or an Euler circuit, or a Hamilton path or Hamilton circuit.}

\bigskip

Below are two graphs. In the table below the graphs, you are asked for each graph whether the graph has an Euler circuit; an Euler path; a Hamilton circuit; or a Hamilton path. Circle YES if the graph has the indicated property and NO otherwise. You do not need to show work or construct the path or circuit. Also, in this problem an ``Euler path'' or ``Hamilton path'' means a path \emph{that is not a circuit}. 


		\begin{center}
			\includegraphics[width=0.4\textwidth]{m20-0a.png} \qquad 
			\includegraphics[width=0.4\textwidth]{m20-0b.png} 		
		\end{center}


\end{document}