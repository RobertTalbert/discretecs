\documentclass{beamer}

\usepackage[utf8]{inputenc}
\usepackage{hyperref}
\usetheme{Warsaw}

%Information to be included in the title page:
\title{MTH 225: Discrete Structures 1}
\subtitle{Module 1, Day 1: Representing integers in different bases}
\author{Prof. Talbert}
\institute{GVSU}
\date{\today}


\begin{document}

\frame{\titlepage}


\begin{frame}{Agenda for today}
    \begin{itemize}
        \item<1-> Review of Daily Prep assignment and Q+A
        \item<2-> Polling activity: Properties of different number bases
        \item<3-> Activity: Converting integers to different bases
        \item<4-> Minilecture: The division/remainder algorithm
        \item<5-> Activity: Working with the algorithm
        \item<6-> Next actions 
    \end{itemize}
\end{frame}

\begin{frame}
    \frametitle{Polling for today}
    \begin{Large}
        \begin{center}
            Go to \url{Mentimeter.com} and enter the code \\ 
            \texttt{XX YY ZZ} 
        \end{center}   
    \end{Large}
\end{frame}

\begin{frame}
    \frametitle{Activity: Converting number bases}

Google Jamboard: An online collaborative whiteboard. Can write on it, or write on paper, snap a photo and upload the photo. 

    \begin{Large}
        \begin{center}
            Go to \url{https://bit.ly/3fWLpkz}
        \end{center}   
    \end{Large}

\end{frame}

\begin{frame}
    \frametitle{An algorithm for making this simpler}

    \begin{block}{Decimal to base $b$ conversion}
        \begin{itemize}
            \item Let $n$ be a (positive) decimal integer and $b$ is the base we're converting to
            \item Let $m$ be the result, initially empty. 
            \item Repeat the following until $n=0$: 
            \begin{itemize}
                \item Divide $n$ by $b$, let $d$ be the quotient and $r$ the remainder 
                \item Write $r$ as the left-most digit of $m$ 
                \item Let $d$ be the new value of $n$ 
            \end{itemize}
        \end{itemize}
    \end{block}

\end{frame}


\begin{frame}
    \frametitle{Example: Convert 3000 to octal (Base 8)}

Start: $n=3000$, $b=8$, $m=$ (empty). 

    \begin{enumerate}
        \item $3000/8 = 375$ remainder $0$. So $d=375$, $r=0$. $m = 0$, new $n$ is $375$ \pause
        \item $375/8 = 46$ remainder $7$. So $d=46$, $r=7$. $m = 70$, new $n$ is $46$ \pause
        \item $46/8 = 5$ remainder $6$. So $d=5$, $r=6$. $m = 670$, new $n$ is $5$ \pause
        \item $5/8 = 0$ remainder $5$. So $d=0$, $r=5$. $m = 5670$, new $n$ is $0$ \pause
        \item STOP because $n$ is now $0$. Result: $3000_{10} = 5670_8$.
    \end{enumerate}

\end{frame}

\begin{frame}
    \frametitle{Activity: Working with the algorithm}

    \begin{Large}
        \begin{center}
            Go to \url{https://bit.ly/3fWLpkz}
        \end{center}   
    \end{Large}


\end{frame}

\begin{frame}
    \frametitle{NEXT TIME...}

    \begin{itemize}
        \item Complete followup activities -- see discussion board and announcements for details 
        \item Go ahead and start Daily Prep for Wednesday (Due Tuesday 11:59pm ET)
        \item Weekly Practice 1: 
    \end{itemize}

Take 5 minutes to fill this form out: https://bit.ly/3a4fexD

\end{frame}

\end{document}