\documentclass[11pt,letterpaper]{article}

\usepackage{fancyhdr}
\usepackage[latin1]{inputenc}
\usepackage{amsmath}
\usepackage{amsfonts}
\usepackage{amssymb}
\usepackage{graphicx}
\usepackage[hmargin=2cm,vmargin=2.5cm]{geometry}
\usepackage[normalem]{ulem}
\usepackage{enumerate}

\newcommand{\workingDate}{\textsc{5 January 2015}}
\newcommand{\courseName}{MTH 410 (Talbert)}
\newcommand{\institution}{Grand Valley State University}

\pagestyle{fancy}
\setlength\parindent{0in}
\setlength\parskip{0.1in}
\setlength\headheight{15pt}

%%%%%%%%%%% HEADER / FOOTER %%%%%%%%%%%
\rhead{\workingDate}
\chead{\textsc{Sample Work Activity}}
\lhead{\textsc{\courseName}}
\rfoot{\textsc{\thepage}}
\cfoot{\textit{Built: \today}}
\lfoot{\textsc{\institution}}

\begin{document}

As a part of working within our competency-based assessment system, we'll be spending time in class discussing what constitutes student work that receives a \textbf{Pass}, \textbf{Progressing}, or \textbf{No Pass} mark, so that you'll be able to judge your own work when the time comes, before handing it in. \\

In today's installment we have examples of student work (all made up by the professor, but realistic) that are solutions to the following problem: 
\begin{quote}
	For every integer $n$, if $n$ is odd then $8$ divides $n^2 - 1$. 
\end{quote}
Make the following assumptions about the context of the student work: 
\begin{itemize}
	\item This problem is part of a Learning Module and is a proof. 
	\item Pretend that this is the ONLY item in the Learning Module. This is totally unrealistic, but we'll explain. 
	\item Each student below submitted their work on time and met all the technical formatting specifications. 
	\item Each student sample below is, verbatim, what the student submitted. 
\end{itemize}


\subsection*{Student sample 1} % (fold)
\label{sub:student_sample_1}

We will show that if $n$ is odd, then $8$ divides $n^2 - 1$. We have the following data: 
\begin{itemize}
	\item If $n = 1$ then $n^2 - 1= 0$ which is divisible by $8$. 
	\item If $n = 3$ then $n^2 - 1 = 8$ which is divisible by $8$. 
	\item If $n = 5$ then $n^2 - 1 = 24$ which is divisible by $8$.
	\item If $n = 7$ then $n^2 - 1 = 48$ which is divisible by $8$. 
\end{itemize}
We can see that the pattern is that whenever $n$ is odd, $8$ divides $n^2-1$. 


% subsection student_sample_1 (end)


\subsection*{Student sample 2} % (fold)
\label{sub:student_sample_2}

\begin{align*}
	n^2 - 1 &= (2k+1)^2 - 1 \\ 
	&= 4k^2 + 4k - 1 + 1 \\
	&= 4k^2 + 4k \\
	&= 4k(k+1)
\end{align*}
Divisible by 8 because either $k$ or $k+1$ is odd, so either $4k$ or $4(k+1)$ is a multiple of $8$.


\subsection*{Student sample 3} % (fold)
\label{sub:student_sample_3}

% subsection student_sample_3 (end)

We will prove this result by mathematical induction. If $n=1$ then $n^2 - 1=0$. This is divisible by $8$ since $0 = 8 \cdot 0$. Therefore the base case holds. 

Now suppose that for some odd integer $k$, we have $8$ divides $k^2 - 1$. We wish to prove that the proposition is true for the next odd integer, which is $k+2$. That is, we wish to show that $8$ divides $(k+2)^2 - 1$. Expanding $(k+2)^2 - 1$ gives: 
\begin{equation}
	(k+2)^2 - 1 = k^2 + 4k + 4 - 1
\end{equation}
Rearrange this to get: 
\begin{equation}
	(k+2)^2 - 1 = k^2 - 1 + 4k + 4
\end{equation}
By the induction hypothesis, $k^2 - 1$ is divisible by 8. It suffices then to show that $4k+4$ is divisible by $8$. To that end, note that $k$ is odd; therefore there exists an integer $j$ such that $k = 2j+1$. Substituting into $4k+4$ gives: 
\begin{align*}
	4k+4 &= 4(2j+1) + 4 \\
	&= 8j + 4 + 4 \\
	&= 8j + 8 \\
	&= 8(j+1)
\end{align*}
Since there exists an integer (namely, $j+1$) such that $4k+4 = 8(j+1)$, we have shown that $4k+4$ is divisible by $8$. Since both $4k+4$ and $k^2 - 1$ are divisible by $8$, we have that thir sum $k^2 - 1 + 4k + 4$ is divisible by $8$, which is what we wanted to prove. 


\subsection*{Student sample 4} % (fold)
\label{sub:student_sample_3}

Assume that $n$ is an odd integer. We want to show that $8$ divides $n^2 - 1$. Since $n$ is odd, there exists an integer $k$ such that $n = 2k+1$. Substituting this into $n^2 -1$ gives
\begin{align*}
	n^2 - 1 &= (2k+1)^2 - 1 \\
	&= 4k^2 + 4k + 1 - 1 \\
	&= 4k^2 + 4k \\
	&= 4k(k+1)
\end{align*}
Since $k$ is an integer, either $k$ is even or $k+1$ is even. In either case, $n^2 - 1$ contains a factor of 8, which is what we wanted to show. 

	

% subsection student_sample_3 (end)


\end{document}