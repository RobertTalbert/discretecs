\documentclass{beamer}

\usepackage{listings}
\usepackage[utf8]{inputenc}
\usetheme{Warsaw}

%Information to be included in the title page:
\title{MTH 201: Calculus}
\subtitle{Module 2A: Intro to modular arithmetic}
\author{Prof. Talbert}
\institute{GVSU}
\date{\today}


\begin{document}

\frame{\titlepage}


\begin{frame}{Agenda for today}
    \begin{itemize}
        \item Review + QA over Daily Prep (Division Algorithm, the \texttt{\%} operator)
        \item Lecture + activity: Writing a program to implement the shift cipher 
        \item Lecture + activity: The multiplicative (decimation) cipher 
        \item Wrap up + feedback
    \end{itemize}
\end{frame}

\begin{frame}
    \frametitle{Q+A from Daily Prep}

    \begin{itemize}
        \item Insert questions here 
    \end{itemize}

\end{frame}

\begin{frame}[fragile]
    \frametitle{Modular arithmetic and the shift cipher}
    
    \begin{enumerate}
        \item User: Input a word to encrypt and a key $K$ (positive integer)
        \item For each letter in the word: 
        \begin{enumerate}
            \item Convert to number $n$ between 0 and 25 
            \item Compute \underline{\hspace{2in}}
            \item Convert back to letter 
        \end{enumerate}
    \end{enumerate}

Now look at a \textbf{Jupyter notebook with some Python code} that does this. 

\begin{center}
    \url{https://bit.ly/2XP25DO}
\end{center}

\end{frame}

\begin{frame}[fragile]
    \frametitle{The multiplicative (decimation) cipher}

    Just like the shift cipher except multiply by the key instead of add. 
    
    \begin{enumerate}
        \item User: Input a word to encrypt and a key $K$ (positive integer)
        \item For each letter in the word: 
        \begin{enumerate}
            \item Convert to number $n$ between 0 and 25 
            \item Compute $(n \times K) \, \% \, 26$
            \item Convert back to letter 
        \end{enumerate}
    \end{enumerate}

Now look at a \textbf{Jupyter notebook with some Python code} that does this. 

\end{frame}

\begin{frame}
    \frametitle{Recapping the highlights}

    \begin{itemize}
        \item The Division Algorithm gives us a way of dividing any two integers and getting a quotient and a remainder 
        \item That remainder is really important in CS and cryptography; \texttt{\%} is an operator that captures 
        \item We can implement all kinds of ciphers in code using \texttt{\%}
        \item However the math gets complicated, for example in finding ``fractions''
    \end{itemize}

\end{frame}


\begin{frame}
    \frametitle{NEXT TIME...}

\begin{center}
    \textbf{All due dates are on the course calendar}
\end{center}


\end{frame}
\end{document}