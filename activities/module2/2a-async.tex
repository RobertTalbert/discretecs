\documentclass[11 pt]{article}

\setlength{\oddsidemargin}{-.35in}
%\setlength{\evensidemargin}{-.5in}
\setlength{\textwidth}{7.0in}
\setlength{\topmargin}{-0.95in}
\setlength{\textheight}{10.0in}

\usepackage{latexsym}
\usepackage{amsfonts}
\usepackage{amsmath}
\usepackage{amsthm}

\usepackage[T1]{fontenc}
\usepackage{baskervald}
\usepackage[bigdelims,vvarbb]{newtxmath} 

\usepackage{hyperref} 
\hypersetup{colorlinks=true, linkcolor=blue,  anchorcolor=blue,  
citecolor=blue, filecolor=blue, menucolor=blue, pagecolor=blue,  
urlcolor=blue,pdftitle={MTH 124 Algebra Expectations}}

\pagestyle{empty}

%\renewcommand{\baselinestretch}{1.5}

\begin{document}

\noindent MTH 201: Calculus \ \ \ Asynchronous lesson activities: Module 2A \hfill Talbert \\


\noindent {\bfseries Directions:} For each item below, give a complete response that represents a good-faith effort to be right. You will receive a "check" if each item has such a response, and an "x" otherwise. An "x" will be given if \emph{any} item is left blank, shows insufficient effort, or has responses such as ``I don't know'' or ``I don't understand''. Except for the final item (which is done by filling out a Google Form), do all work on separate pages, and submit a scanned black/white PDF to Blackboard. 

\noindent \hrulefill

\begin{enumerate}

    \item Using $a = 20201970$ and $b = 15$, find the values of $q$ and $r$ guaranteed by the Division Algorithm. Show your work/explain your steps. 

    \item Calculate the following and explain your reasoning on each one. 
    \begin{enumerate}
        \item $245 \, \% \, 8$
        \item $(-245) \, \% \, 8$
        \item $8189123 \, \% \, 2$ (Hint: The explanation here is short.) 
    \end{enumerate}
	
	\item Encrypt the word \texttt{DISCRETE} using the shift cipher and a key of $18$. Show your work/explain your steps. 
	
	\item The coded message \texttt{vdapztgh} was intercepted, and your sources tell you it was encrypted using a shift cipher with a key of 15. Decrypt to find the plaintext message. Show your work/explain your steps. 
	
	\item You've also intercepted the message \texttt{twcqmqakwwt}, and your sources tell you it was encrypted using a shift cipher, but they were unable to get the key. Decrypt to find the plaintext message. Show your work/explain your steps. 
	
	\item Read in the lecture notes (or watch the class recording) that explains how the shift cipher can be implemented mathematically using the modulus operator. The \emph{multiplicative} or ``decimation'' cipher works like the shift cipher, except instead of adding the key to the number representation of a character and reducing the result mod 26, we mutliply by the key and reduce. For example to encrypt the letter \texttt{e} with a key of 7, we convert \texttt{e} to its position in the alphabet (4), multiply by the key ($4 \times 7 = 28$) then reduce mod 26 ($28 \, \% \, 26 = 2$) which is the letter \texttt{c}. Encrypt the message \texttt{DISCRETE} using a multiplicative cipher and a key of 11. 
	
	\item The multiplicative cipher moves letters around in the alphabet, like a shift cipher. But, do all the letters in the plaintext move the same distance? Explain. 
	
	\item Try encrypting \texttt{DISCRETE} using a multiplicative cipher and a key of 13. What do you notice? If you were the recipient of this coded message and you had the key, would you be able to decrypt? Explain.  
	
	\item You've received the coded message \texttt{eapl}, and your sources tell you it was encrypted with a multiplicative cipher using a key of 9. How would you go about decrypting it? What mathematical issues do you run into? 
	
	\item Fill out the feedback form here: \url{https://bit.ly/2XPi1pw}. 
	
\end{enumerate}


\end{document}
