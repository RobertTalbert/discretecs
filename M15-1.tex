\documentclass[11pt]{article}

\pagestyle{empty}                       %no page numbers
\thispagestyle{empty}                   %removes first page number
\setlength{\parindent}{0in}               %no paragraph indents

\usepackage{fullpage}
\usepackage[tmargin = 0.5in, bmargin = 1in, hmargin = 1in]{geometry}     %1-inch margins
\geometry{letterpaper}                  
\usepackage{graphicx}
\usepackage{amssymb}

% Default packages
\usepackage{latexsym}
\usepackage{amsfonts}
\usepackage{amsmath}
\usepackage{amsthm}
\usepackage{palatino}
\usepackage{hyperref}
\usepackage{multicol}
\usepackage{fancyhdr}
\usepackage{pseudocode}

\begin{document}
	
	\thispagestyle{empty}
	\renewcommand{\headrulewidth}{1pt}
	\thispagestyle{fancy}
	\lhead{Prof. Talbert}
	\chead{MTH 325}
	\rhead{M.15 v1}
	\lfoot{}
	\cfoot{}
	\rfoot{}	
	
	\vspace*{0in}

\noindent 
\textbf{M.15: Use the Handshaking Theorem and Theorem 2 to draw conclusions about edges and nodes in a graph.}

\bigskip

Work these two mini-problems: 

\begin{enumerate}
 	\item Two undirected graphs have 20 nodes each, and they have the same degree sequence, namely: 
 	$$[5, 1, 5, 3, 3, 4, 3, 3, 3, 2, 7, 6, 2, 4, 4, 7, 4, 2, 2, 4]$$
 	Must these two graphs have the same number of edges? If you think so, write ``Yes'', then find the number of edges each graph has, and show your work as to how you found that number of edges. If you think not, write ``No'' and then explain or give a counterexample. (Note: Both the answer and the reasoning must be correct and clear in order to Pass.)

\vspace{3in}


 	\item Below is a list of 30 integers. Does a graph exist that has this list as its degree sequence? Clearly state ``yes'' or ``no'' and then explain your reasoning briefly. (Note: Both the answer and the reasoning must be correct and clear in order to Pass.)
 	$$[3, 2, 5, 2, 2, 0, 0, 4, 3, 4, 1, 3, 2, 5, 3, 1, 3, 6, 4, 3, 2, 4, 2, 3,
2, 2, 6, 1, 3, 4]$$ 	
 \end{enumerate} 

\medskip

% Use the space below and on the back for your answers and work, and clearly label which answer and work goes with which part. 




\end{document}