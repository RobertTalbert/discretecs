\documentclass[]{article}
\usepackage{palatino}
\usepackage{fullpage}
\usepackage{longtable}
\usepackage{makecell}
\usepackage{tcolorbox}
\usepackage{multirow}
\usepackage[table,xcdraw]{xcolor}
\usepackage{float}
\usepackage[urlcolor=blue]{hyperref}
% \hypersetup{
%     linkcolor=blue,
%     urlcolor = blue
% }
\usepackage{amssymb,amsmath}
\usepackage{ifxetex,ifluatex}
\usepackage{graphicx}
\usepackage{fixltx2e} % provides \textsubscript
\ifnum 0\ifxetex 1\fi\ifluatex 1\fi=0 % if pdftex
  \usepackage[T1]{fontenc}
  \usepackage[utf8]{inputenc}
\else % if luatex or xelatex
  \ifxetex
    \usepackage{mathspec}
    \usepackage{xltxtra,xunicode}
  \else
    \usepackage{fontspec}
  \fi
  \defaultfontfeatures{Mapping=tex-text,Scale=MatchLowercase}
  \newcommand{\euro}{€}
\fi
% use upquote if available, for straight quotes in verbatim environments
\IfFileExists{upquote.sty}{\usepackage{upquote}}{}
% use microtype if available
\IfFileExists{microtype.sty}{%
\usepackage{microtype}
\UseMicrotypeSet[protrusion]{basicmath} % disable protrusion for tt fonts
}{}
\ifxetex
  \usepackage[setpagesize=false, % page size defined by xetex
              unicode=false, % unicode breaks when used with xetex
              xetex]{hyperref}
\else
  \usepackage[unicode=true]{hyperref}
\fi
\usepackage[usenames,dvipsnames]{color}
\hypersetup{breaklinks=true,
            bookmarks=true,
            pdfauthor={},
            pdftitle={},
            colorlinks=true,
            citecolor=blue,
            urlcolor=blue,
            linkcolor=magenta,
            pdfborder={0 0 0}}
\urlstyle{same}  % don't use monospace font for urls
\setlength{\parindent}{0pt}
\setlength{\parskip}{6pt plus 2pt minus 1pt}
\setlength{\emergencystretch}{3em}  % prevent overfull lines
\providecommand{\tightlist}{%
  \setlength{\itemsep}{0pt}\setlength{\parskip}{0pt}}
\setcounter{secnumdepth}{0}

\date{}

% Redefines (sub)paragraphs to behave more like sections
\ifx\paragraph\undefined\else
\let\oldparagraph\paragraph
\renewcommand{\paragraph}[1]{\oldparagraph{#1}\mbox{}}
\fi
\ifx\subparagraph\undefined\else
\let\oldsubparagraph\subparagraph
\renewcommand{\subparagraph}[1]{\oldsubparagraph{#1}\mbox{}}
\fi

\begin{document}


\begin{flushleft}
  \includegraphics[width=\textwidth]{225-banner-2.png}
\end{flushleft}


\textbf{Welcome to MTH 225!} I'm Robert Talbert, Professor of Mathematics and the instructor for the course, and I am glad you're here. 

In MTH 225, you'll be learning \textbf{the math that computer science is built on}. You'll learn things like how to do arithmetic in binary, how to count the number of ways to deal a five-card poker hand, and how to generate complex data structures using simple rules involving recursion. And more! By studying discrete structures, \textbf{you'll gain a superpower to make you an expert learner} of any hardware and any software, including those that haven't been invented yet! 

\textbf{My \#1 job as the professor is to make sure you succeed in learning.} Success in the course doesn't come cheap. Like learning to program, learning math involves hard work, a willingness to try things that might not work the first time, and the ability to improve using feedback. \textbf{But I promise you that I will work to make MTH 225 a class where you can make mistakes and grow safely and productively.} 

\textbf{Catalog description}: \textit{Introduction to the mathematical foundations of computer science. Topics to include: number bases and modular arithmetic; symbolic logic, propositions, and quantification; sets, set operations, and functions; combinatorics and combinatorial proof; recurrence relations and mathematical induction.} 

\begin{tcolorbox}[title=Course Information, colback=green!15!white]


\begin{itemize}
\tightlist
    \item \textbf{Meetings}:  Section 01 meets MWF 10:00-10:50am in MAK A2155. Section 02 meets MWF 11:00-11:50am in MAK BLL118. All meetings are conducted in-person. 
    \item \textbf{Course calendar:} The official course calendar is a Google Calendar, linked on the Blackboard site in the \textit{Calendar} area. This calendar  contains all date-sensitive events such as due dates and is always considered to be correct in the event of apparent date conflicts. 
    \item \textbf{Definition of ``week''}: In our course, a ``week'' is defined to begin at 12:01am Eastern time on Sunday, and end at 11:59pm Eastern time the following Saturday. 
    \item \textbf{Instructor}: Robert Talbert, Ph.D., Professor of Mathematics (office: MAK C-2-513)
    \item \textbf{Instructor contact}: On Campuswire via direct message, or by email at \url{talbertr@gvsu.edu}. Phone: 317-331-8968. 
    \item \textbf{Drop-in (office) hours}: To be determined in week 1 by student input.
    \item \textbf{Availability}: \textit{I check messages only between 6am and 6pm on weekdays, and once on weekends}. Messages received during those times that need a reply will receive one in 6 hours or less. Otherwise replies will be sent when back online. If you have a question that can be asked publicly, you will get a faster response if you post it to Campuswire.
    \item \textbf{Textbook:} Some content in the course will use the textbook \href{http://discretetext.oscarlevin.com/dmoi/}{Discrete Mathematics: An Open Introduction} by Oscar Levin, which is available for free online. If you prefer a paper copy, \href{https://www.amazon.com/gp/product/1516921186}{you can purchase one from Amazon} (although note this is an earlier edition). The book is not sold in the GVSU bookstore. 
    \item \textbf{Videos:} Found at this Vimeo playlist:  \href{https://vimeo.com/showcase/8667148}{https://vimeo.com/showcase/8667148}
    \item \textbf{Required technology}: Please bring a \textbf{laptop or tablet device with a reliable internet connection to each class meeting} so you can participate in group activities and stay connected with email, Blackboard, and the discussion board. Smartphones are OK but not recommended. If you do not have a device, please contact me to discuss your situation. We'll use a variety of software tools that are all free web applications. 
\end{itemize}
\end{tcolorbox}

\vfill \eject

\section{What will I learn in MTH 225?}

MTH 225 is the first of a two-semester sequence on \textit{discrete mathematics} for computer science. What's ``discrete mathematics''? Basically, it's \textbf{mathematics applied to situations that involve things that can be separated and counted}. For example, counting the number of times a loop in a computer program executes involves separating things (the different iterations of the loop) and counting them.  \textbf{So in MTH 225 we look at the \textit{mathematical processes} that computer science is built on, especially the \textit{structures} that are the basis for the data structures you'll encounter later.} 

\textbf{Course-level learning objectives:} Upon completion of MTH 225, you will be able to:
\begin{itemize}
\tightlist
    \item Represent integers using different number bases, and perform integer arithmetic using different bases and modular arithmetic.
    \item Formulate, manipulate, and determine the truth of logical expressions using symbolic logic.
    \item Formulate and solve computational problems using sets and functions.
    \item Formulate and solve complex counting problems using computational thinking and the tools of combinatorics.
    \item Evaluate numerical and other sequences using recursion, and solve simple recurrence relations.
    \item Write clear, correct, and convincing arguments to explain the correctness of a solution using combinatorial proof and mathematical induction.
    \item Explain the reasoning behind solutions to computational problems clearly to an appropriate audience.
    \item Apply effective problem-solving skills in solving computational problems.
    \item Apply computer programming and computational thinking to frame and solve mathematical and computational problems.
    \item Self-assess one's work and apply feedback from others to make improvements in that work.
\end{itemize}

\textbf{Course module structure:} The course content is split up into five \textbf{modules}:
\begin{itemize}
\tightlist
    \item \textbf{Module 1: Computer arithmetic}. Representing integers in binary, octal, and hexadecimal; binary arithmetic; the Division Algorithm and modular arithmetic. 
    \item \textbf{Module 2: Logic}. Logical propositions, conditional statements, truth tables, predicates, and quantification. 
    \item \textbf{Module 3: Sets and functions}. Set notation and representation, set operations, functions (including special computer science functions). 
    \item \textbf{Module 4: Combinatorics}. The Additive and Multiplicative counting principles, the binomial coefficient, permutations, stars-and-bars counting methods. 
    \item \textbf{Module 5: Recursion and induction}. Numerical sequences in closed-formula and recursive forms, solutions to recurrence relations, the Principle of Mathematical Induction and proof by induction. 
\end{itemize}
A visual structure of these modules can be found on the course Blackboard site in the \textit{Course Documents} area.  

\textbf{Learning Targets:} The basic skills of the course are given as 20 \textbf{Learning Targets}. These are listed in \hyperref[sec:learning-targets]{Appendix B}. Eight of these are designated as \textbf{Core} targets, meaning they are the most essential skills in the course. As detailed next, a major goal for you in the course is to demonstrate fluency on as many of these learning targets as possible through various kinds of evidence.  

\begin{tcolorbox}[colback=yellow!15!white]
In everything we learn in the course, we seek a \textbf{conceptual understanding from multiple perspectives}, the ability to \textbf{apply foundational ideas to new situations}, development of \textbf{logical reasoning and communication skills}, and the ability to \textbf{use feedback to improve your reasoning and problem solving skills}. While there's value in being able to get the right answers to computations, there's more value in showing mastery of the concepts involved, and that's where your work in the course will be focused.}
\end{tcolorbox}


\section{What will I do to learn?}

\begin{tcolorbox}[title=About grades and assignments in MTH 225, colback=yellow!15!white]
MTH 225 uses a \textbf{mastery-based grading system}. What this means is: 
\begin{itemize}
\tightlist
    \item \textbf{Most graded work will not have point values attached}. Instead, your work is evaluated on a \textbf{pass/no pass basis}, depending on whether or not it meets predefined standards. 
    \item Since there are no points, \textbf{there is no partial credit}. Instead, most work that does not meet the standards will be given \textbf{helpful feedback}, and can then be \textbf{revised and resubmitted to improve the work} until it does meet the standard. 
\end{itemize}
This may be different than what you are used to, and you might need some time to adjust. That's OK, and I'm committed to providing you with information and tools to help you thrive in this system. I've been using mastery-based grading in all my courses since 2015 and it has led to dramatic improvements in student work and stress levels because \textbf{it gives you the time and space to make mistakes and learn from them, rather than relying on point-focused one-and-done testing}. \\

More information is below, and more will be forthcoming in week 1. In particular, the separate document \textbf{``Specifications for Satisfactory Work in MTH 225''} contains details about what constitutes ``passing'' work on each of the assignments below. You can find this on Blackboard in the \textit{Course Documents} area and also \href{https://hackmd.io/@rtalbert235/SyVUNKrkt}{by clicking here}.
\end{tcolorbox}

In MTH 225, you'll be completing a variety of assignments designed to get you \textbf{actively involved in the learning process} and \textbf{provide evidence of mastery of the course concepts}.

\begin{description}
\item[Daily Prep.] These are assignments that \textit{prepare you for class work}. In them, you will get basic fluency on new concepts through guided independent work. Daily Prep assignments have two parts.
    \begin{itemize}
    \tightlist
        \item \textit{Pre-class}: Before class, you'll watch videos at the \href{https://vimeo.com/showcase/8667148}{course  playlist}, complete short readings, and then work  exercises over the material. These are graded 1 point each on the basis of completeness and effort. 
        \item \textit{In-class}: At the beginning of class, you'll pair/triple off with classmates to complete a short quiz over the material from the pre-class portion. These are graded 1 point each on the basis of overall correctness. 
    \end{itemize}
\item[Weekly Challenges.] These are weekly packages containing practice exercises, application problems, and feedback/reflection prompts. They are graded either \textbf{Satisfactory} or \textbf{Unsatisfactory} on the basis of overall completeness, effort, and quality. 
\item[Learning Targets.] You'll have the opportunity to demonstrate fluency on the 20 Learning Targets in the course. You can do so through a combination of any of the following: 
    \begin{itemize}
    \tightlist
        \item \textit{Quizzes}: Several class meetings are set aside to take written quizzes over Learning Targets. Each problem on a quiz focuses on just one Learning Target and evaluated for overall correctness of the answers and explanations.  
        \item \textit{Oral quizzes}: You can come to drop-in hours or schedule a video call to do an oral quiz over a Learning Target. You tell me which Learning Target you want to try, and I'll make up a quiz-like problem for it that you then do ``live''. These are evaluated the same as written quizzes. 
        \item \textit{Videos}: You can also make a video of yourself working out a quiz-like problem at a whiteboard. These are set up and evaluated the same as oral quizzes.
        \item \textit{Something else}: If you have some other means you want to use to demonstrate your skill on a Learning Target, tell me about it and we'll discuss how we might use it as evidence. 
    \end{itemize}
Any of these done at a \textbf{Satisfactory} level will be considered a ``successful demonstration of skill'' on the Learning Target. \textbf{Once you have provided two (2) successful demonstrations of skill on a Learning Target, you are certified as Fluent on that target.} On Blackboard, each Learning Target has its own entry with either 0, 1, or 2 listed in it, for the number of successful demonstrations of skill you have so far provided. Once you have reached fluency on a Learning Target by giving 2 successful demonstrations of skill, no more work on that Learning Target is required.  
\end{description}

\begin{tcolorbox}[title=Examples, colback=yellow!15!white]
\begin{enumerate}
    \item Alice wants to become fluent on Learning Target L.1. In week 4, she attempts a quiz problem over this Learning Target and does \textbf{Satisfactory} work, and her gradebook entry will show a ``1''. Then, in week 6, she takes another quiz and does \textbf{Satisfactory} work again on that target. Those two demonstrations of skill mean she is ``fluent'' L.1. Her gradebook entry now shows ``2'' to indicate this, and no further demonstrations are needed in the course.  
    \item Bob also wants to master Learning Target L.1 and attempts the quiz in week 4, but his work doesn't meet the standards. His gradebook entry remains at ``0'' for that target. So he practices at home, and the following week he comes to drop-in hours and asks for an oral quiz; I make up a problem similar to the written quiz and he does \textbf{Satisfactory} work on it, so he's upgraded to a ``1'' on Blackboard. Then in week 6, he attempts a written quiz problem on it and does \textbf{Satisfactory} work on that too. His Blackboard record shows ``2'' meaning Bob is now ``fluent'' on L.1 --- just like Alice, but Bob did it with a different package of evidence. 
\end{enumerate}
The purpose of this system is to allow you flexibility in how you demonstrate skill on Learning Targets. As long as you can do so twice, \textit{how} you do it doesn't matter so much. 
\end{tcolorbox}    
    
\textbf{Final Exam:} There will be a final exam for the class on \textbf{Monday, December 13, 10:00-11:50am for Section 01 and Wednesday, December 15, 10:00-11:50am for Section 02}. The final exam is in two parts. Part 1 is a comprehensive assessment covering the entire course, consisting of both mathematical and reflection/essay questions. Part 1 is graded on a 0--100 scale. Part 2 is optional and consists of a final quiz problem for any Learning Target on which you haven't become fluent by that point. 

\section{How will I know if I am learning? (A.K.A. How are things graded?)}

\textbf{How individual items are graded:} Individual assignments are graded in different ways, generally on a \textbf{two-level scale}, either Satisfactory/1 point or Unsatisfactory/0 points. The specifications for what constitutes ``Satisfactory'' are detailed in the separate document \textbf{``Specifications for Satisfactory Work in MTH 225''} found on Blackboard in the \textit{Course Documents} area and also \href{https://hackmd.io/@rtalbert235/SyVUNKrkt}{by clicking here}.
 

\textbf{How the course grade is determined:} Your \textit{base grade} in the course (A,B,C,D, or F without plus/minus modifiers) is determined using the following table. \textbf{To earn a grade, meet or exceed all the requirements in the column for that grade}: 

\begin{center}
    \begin{tabular}{c|l}
 To earn: & Accomplish the following: \\ \hline \hline
 \textbf{A} & \makecell[l]{Earn at least \textbf{45 points on Daily Prep assignments};
\emph{and} \\ Attain fluency on \textbf{all 8 Core Learning Targets plus any 10 others}; \emph{and} \\ Earn Satisfactory marks on \textbf{at least 10 Weekly Challenges}.} \\ \hline 
 \textbf{B} & \makecell[l]{Earn at least \textbf{40 points on Daily Prep assignments};
\emph{and} \\ Attain fluency on \textbf{all 8 Core Learning Targets plus any 8 others}; \emph{and} \\ Earn Satisfactory marks on \textbf{at least 9 Weekly Challenges}.} \\ \hline 
 \textbf{C} & \makecell[l]{Earn at least \textbf{35 points on Daily Prep assignments};
\emph{and} \\ Attain fluency on \textbf{all 8 Core Learning Targets plus any 6 others}; \emph{and} \\ Earn Satisfactory marks on \textbf{at least 8 Weekly Challenges}.} \\ \hline 
 \textbf{D} & \makecell[l]{Earn at least \textbf{25 points on Daily Prep assignments};
\emph{and} \\ Attain fluency on \textbf{Any 6 Learning Targets} (not necessarily Core); \emph{and} \\ Earn Satisfactory marks on \textbf{at least 4 Weekly Challenges}.} \\ \hline 
\end{tabular}
\end{center}

A grade of \textbf{F} is given if not all of the requirements for a \textbf{D} are met. 

\textbf{Fluency:} As described earlier, ``attaining fluency'' on a Learning Target requires \textbf{two successful demonstrations of skill} on that target through a combination of evidence. 

\textbf{Number of assignments:} The initial plan is to have \textbf{25 Daily Prep assignments }(for a total of 50 points) and \textbf{11 Weekly Challenges}. Note that this means \textbf{nobody has to do every single assignment given}. If we have more than these, you can use the extras as ``extra credit''. If we have fewer, we'll adjust the table above.

\textbf{Plus/minus modifiers:} If you score a 80\% or higher on the comprehensive portion of the final exam, you will earn a ``+'' on your base grade. If you score a 60\% or lower, you will earn a ``-'' on the base grade. The final exam has no other effect on your course grade. I may add a plus or minus to a course grade for other reasons, at my discretion. In particular, if you are close to completing all the requirements for a grade, I will take your situation under consideration and possibly give a plus or minus. 

Please note, Grand Valley State does not award grades of A+ or D-. 



\subsection{Feedback and revision}

\begin{tcolorbox}[title=Feedback loops in MTH 225, colback=yellow!15!white]
All humans learn through \textbf{feedback loops} where we try something, make mistakes, get feedback on those mistakes, and then incorporate the feedback into a new version of what we tried. Feedback loops are at the heart of how this course works. \\

Except for Daily Prep and the Final Exam, \textbf{you are not penalized for unsatisfactory work.} Instead, work that doesn't meet the standards for \textbf{Satisfactory} will be given \textbf{helpful feedback} about what was good and what needs improvement, and then you can \textbf{reattempt it later with improvement} and \textbf{keep repeating this feedback loop} until your work is \textbf{Satisfactory}. (Within certain restrictions; see below.) 
\end{tcolorbox}

Weekly Challenges and Learning Target attempts that do not meet the standards for \textbf{Satisfactory} work can be revised and/or reattempted as follows: 

\textbf{To reattempt a Weekly Challenge}: Weekly Challenges are submitted electronically on Blackboard. For Weekly Challenges marked \textbf{U} (\textbf{Unsatisfactory}), feedback will be added directly on to the electronic submission. To reattempt, simply read through the feedback, complete a \textbf{Feedback Response Form} (posted in the \textit{Forms} area on Blackboard), and then upload a new version of the Weekly Challenge with corrections. 

\textbf{If work on a Weekly Challenge is marked IC (Incomplete)}: Work marked \textbf{IC} is not considered a good-faith attempt at a submission because of significant or persistent errors, omissions, or issues with mathematics or writing. Work that is marked \textbf{IC} will be given minimal or no feedback, and you must spend a token (see below) to reattempt it. You can easily avoid receiving an \textbf{IC} by reviewing the \textbf{``Specifications for Satisfactory Work in MTH 225''} document on Blackboard in the \textit{Course Documents} area before submitting. 

\textbf{To reattempt a Learning Target}: If you attempt work on a Learning Target that doesn't meet the standards, you'll be given feedback on your work. To reattempt, simply complete a \textbf{Feedback Response Form} (posted on Blackboard in the \textit{Forms} area) and then use any of the methods mentioned in this syllabus for demonstrating fluency on that target at a later date. For example, you can attempt a quiz problem at a later date; or take an oral quiz in office hours on a video call; or make a video. 

\textbf{Revisions on other work:} Due to the time-bound nature of Daily Prep assignments and the Final Exam, unfortunately no revisions are possible on those. 

\textbf{Restrictions on reattempts}: In order to keep the logistics of grading reasonable and to help you get the most out of your revisions, there are some limitations on this revision policy: 
\begin{itemize}
\tightlist
    \item \textbf{Weekly Challenges can only be reattempted }\textbf{twice} (for a total of three attempts). 
    \item \textbf{No more than two Weekly Challenge reattempts can be submitted in a single week} (without spending a token; see below).
    \item There are deadlines on the course calendar for revisions of Weekly Challenges. No revisions past those deadlines are accepted.  
    \item \textbf{All reattempts must be accompanied by a Feedback Response Form in order to be accepted}. If you're submitting multiple reattempts, you can use just one form. You can substitute the form with a face-to-face or video conversation if you like. The main thing is that you need to spend time critically thinking about your work before reattempting it. 
\end{itemize}

\textbf{Tokens}: Tokens are a ``currency'' in the course that you can use to purchase exceptions to the course rules. You begin the course with five tokens in your ``account'' (found as a column in the Blackboard gradebook), and one token can purchase any of the following:
\begin{itemize}
\tightlist
    \item A 24-hour extension on any deadline; 
    \item One point on Daily Prep (\textit{limit 3 for the semester});
    \item A third submission of a Weekly Challenge revision in a given week;
    \item Revision of a Weekly Challenge marked \textbf{IC}; 
    \item An additional untimed Daily Prep quiz makeup after the first three; 
    \item Any other bending of the course rules you might want --- just discuss with me first. 
\end{itemize}
To spend a token, fill out a \textbf{Token Spending Form} found on Blackboard in the \textit{Forms} area. Once you have submitted the form, the item you ``purchased'' is yours --- you don't need to wait for me to give permission or to respond to you. Token totals will be updated weekly. 


\section{What can I do to help myself learn?}

The feedback loops you use to learn extend beyond the class meetings in several key ways. Here are some ways to get help of various kinds. 


\textbf{Campuswire}: The class discussion board is the default location for all public-facing questions about the course or the material. If you have a question that can be asked in public and which might benefit others, \textbf{ask it on Campuswire first}. I'll receive a notification, so not only will I see your question, so will others, and you'll get an answer at least as quickly if you just emailed me. You can ask questions anonymously on Campuswire if it helps. 

\textbf{Drop-in hours}: We will determine drop-in hours during the first week of classes. You do not need an appointment for drop-in hours; just show up with your questions ready. 

\textbf{Appointments:} I will be available for scheduled appointments either in person on through video calls, outside of drop-in hours. Available appointment hours will be given once we set drop-in hours in week 1. 

\textbf{Math Center}: (Waiting for new boilerplate for F2021) 
% GVSU’s Math Tutoring Center is online with Blackboard Collaborate this semester. You can access virtual drop-in tutoring through a link in your Blackboard course called Math Tutoring Center or on our website at http://gvsu.edu/tutoring/math/. Then you need to click on the “Online Math Tutoring Center” button, which will require a GVSU login. There will be tutors online, ready to help, Monday through Thursday 10a – 9p, Friday 10a – 2p and Sunday 6p – 9p starting Wednesday, September 2nd. Bring questions about your technology, on methods and concepts, or on specific problems. All Math Center tutoring is free, so stop by early and often. When you enter the Collaborate room, please type your first and last name so you can get you signed in and connected with a tutor.

\textbf{Special learning needs}: If you have special needs because of learning, physical or other disabilities, it is your responsibility to contact Disability Support Resources (DSR) at 616-331-2490 or \url{http://www.gvsu.edu/dsr/}. DSR will help you arrange accommodations. Then, speak with me in person about making those accommodations and ensure that they are consistent with your arrangements with DSR.

\textbf{Basic needs security}: If you have difficulty affording groceries or accessing sufficient food to eat every day, or if you lack a safe and stable place to live, I encourage you to visit Replenish, a food resource for GVSU students. If you are comfortable doing so, please speak with me about your circumstances so that I can advocate for you and to connect you with other campus resources.

\textbf{Gender identity and expression}: If, for purposes of gender identity and expression, your official name (in Banner) does not match your preferred name, your name can be updated in Blackboard. Please contact the registrar's office to submit this request. The registrar's office will contact the Blackboard administrator to make the change and will also contact your professors to inform them that your name in Banner will not match the name in Blackboard.

\section{What else do I need to know?}

\subsection{Academic integrity}

The university’s academic integrity policy \href{https://www.gvsu.edu/osccr/academic-misconduct-policies-and-procedures-14.htm}{is described  here}. Additionally, the School of Computing and Information Systems has its own guidelines for academic integrity here: \url{http://www.cis.gvsu.edu/academic-honesty/}. \textbf{Every student has the responsibility of reading and understanding these policies, especially the consequences for engaging in academically dishonest activities}.

We value collaboration in MTH 225, but we also value independent mastery of the course concepts. The particular balance of collaboration and independent work depends on the assignment, specifically: 

\begin{itemize}
\tightlist
    \item \textbf{Daily Prep}: You may collaborate with other students on the pre-class portion of Daily Prep, as long as your responses are given in your own words and reflect your own understanding. You \textbf{may not} collaborate with other students on the in-class quiz portion of Daily Prep with the exception of your quiz partner(s). 
    \item \textbf{Weekly Challenges}: You may discuss high-level concepts on Weekly Challenges with other students, and you may seek help in print and online sources, but the work you submit (the answers, supporting work, and explanations) must be your own and reflect your own understanding. No significant portion of your work can be the result of someone else's efforts. This includes any work that is found online, for example on StackExchange, Reddit, or Chegg. 
    \item \textbf{Learning Target work}: You may not collaborate with another person during a quiz, including oral quizzes. If you opt to make a video for a Learning Target, you will be required to keep your face, voice, and handwriting in the frame at all times and not have another person present to coach you. 
\end{itemize}

Please keep in mind that \textbf{there is no need to commit academic dishonesty in the course because of the opportunity to revise and reattempt work or to spend tokens}. If the best you can do on an assignment is a complete, good-faith effort that has a lot of flaws in it, then that's fine! You won't be penalized, and through the feedback loops we have, you can work with it and my feedback to make it good. 


\subsection{Attendance, missed work, and late work}

\textbf{Attendance is expected at all class meetings.} Our meetings will be focused on more difficult concepts, and missing a class meeting will deprive you of activity that is necessary for understanding. I will be recording attendance and reaching out to students who show patterns of absence. 

\textbf{If you must be absent:} You do not need prior permission or justification. A heads-up is appreciated but not necessary. Otherwise it's none of my business.

\textbf{If you miss a Daily Prep group quiz due to lateness or absence:} If you are late for or absent from a class meeting and miss the in-class group quiz portion of Daily Prep, you may request an untimed makeup quiz by submitting a \textbf{Daily Prep Makeup Form} found in the \textit{Forms} area on Blackboard. Such requests must be made \textbf{within 24 hours of the original Daily Prep quiz}; requests after that window closes will be declined. You may make \textbf{up to three makeup requests without penalty} and there are no questions asked and no explanations needed. After the third makeup, \textbf{you must spend a token for any additional makeups}. 

\textbf{If you miss other class work due to absences}: If you miss a Learning Target quiz, there are no makeups allowed, because you can just take it again the next time it's offered, or use an alternative to quizzing such as an oral quiz or video. All other work in the class is done outside of class meetings, so absences won't affect those. 

\textbf{Late work:} The only assignments with deadlines are the out-of-class portion of Daily Prep, and Weekly Challenges. If you can't turn in work on those by the deadline, you can spend a token to extend the deadline by 24 hours. If you are in a situation where you absolutely need more than a 24-hour extension, please let me know as soon as possible and we'll talk. Work that is turned in late without a token and without checking in with me, will not be accepted. 

\textbf{Special case with Weekly Challenges}: Weekly Challenges must be submitted by their deadline or an instructor-approved extended deadline in order to be revised and resubmitted. Without an initial on-time submission there is no ``reattempt'' possible. 

\subsection{Covid-19 related policies}

LEAVE BLANK UNTIL SEMESTER STARTS

\subsection{Tech support}

Technology issues will happen with almost 100\% certainty, usually at the worst possible moment. \textbf{When this happens, do not contact me first --- instead, contact the appropriate office listed below.} I am generally of no help with tech problems since I lack the skill and permissions needed to fix anything. I suggest checking with a classmate as well to make sure the problem is not isolated to just you; and if it's a network problem, wait five minutes and try again before reporting. 


\begin{tcolorbox}[colback=yellow!15!white]
\begin{itemize}
\tightlist
    \item For help with \textbf{Blackboard}: Email the \textbf{Blackboard Help Des}k at \url{bbadmin@gvsu.edu} or call (616) 331-8526. For hours of operation and more information see \url{https://www.gvsu.edu/elearn/help/}.
    \item For help with the \textbf{GVSU network, email, or printing}: Email the \textbf{GVSU IT Help Desk} at \url{helpdesk@gvsu.edu}; or call (616) 331-2101 or toll free (855) 435-7488. For hours of operation and more information see \url{https://www.gvsu.edu/it/}.
    \item For specific help with \textbf{your computer}: Try the \textbf{GVSU IT Help Desk} (see previous bullet) or contact your equipment manufacturer or computer store.
    \item For help with \textbf{course tools}: Ask a question on \textbf{Campuswire}, seek out the \textbf{help documentation} in the tool, or do a \textbf{targeted Google search}.
\end{itemize}
\end{tcolorbox}

\textbf{If tech issues prevent turning in work:} If you have contacted an appropriate source of help and an issue still persists that prevents you from turning in work in the usual way (for example on Blackboard), \textbf{you are expected to take alternative measures to get your work turned in on time.} For example, if Blackboard is truly offline and a deadline is looming, send an email with an attachment; if email isn't working, post a photo of your work on Campuswire in a direct message. Then, submit the work using the normal means later. 

\bigskip

\begin{tcolorbox}[title=Disclaimer, colback=yellow!15!white]
Changes to this syllabus may occur during the semester. In those cases,
the changes will be announced in class and online, and if appropriate,
students will be given a voice on how the changes will be implemented.
Again it is your responsibility to attend class and process all the
information passed along in course announcements so that you will be
aware of any changes that take place.
\end{tcolorbox}



\section{About the instructor}

I'm Robert Talbert, and I'm a Professor in the Mathematics Department.  This is my tenth year at GVSU and my 28th year of teaching overall (not counting tutoring gigs in high school and college). I have Ph.D. in Mathematics from \href{http://www.vanderbilt.edu}{Vanderbilt University} and went to college at \href{http://www.tntech.edu}{Tennessee Technological University}. 

I was, at best, a thoroughly average math student until my senior year of high school, when I had a teacher for Calculus (hi, Mrs. Allen) who stopped trying to cram things into my
head and let me figure concepts out on my own instead. After a two-year
stint as a Psychology major at Tennessee Tech, I changed my major to
Mathematics on a dare from my roommate (long story), and to my great surprise fell in love with the subject. 

I ended up doing a Ph.D. dissertation on an obscure area at the
intersection of abstract algebra and geometry, and also discovered I love teaching math to college students. So I went on to spend 14 years teaching in small liberal arts colleges before coming to GVSU in 2011.  Now I teach computer scientists and engineers how to think like mathematicians,
and I do research on how to make teaching and learning better with
technology and active learning techniques, all while trying to improve my Python skills, which is easy since I'm not that great to begin with, and maybe pick up a little \href{https://www.haskell.org/}{Haskell} and \href{https://julialang.org/}{Julia} on the side. 

I live in Allendale with my wife, three teenage kids, and four cats. I am a decent cook and longtime bass guitar player, and I aspire to spend more time on a bicycle or in a kayak than in front of a computer. You can read more of what I'm thinking about at my website, \href{https://rtalbert.org}{rtalbert.org}. I'm writing a book right now with my GVSU colleague Dr. David Clark about alternative grading practices, and you can read more about that at my other blog, \href{http://gradingforgrowth.substack.com}{Grading For Growth}. 

\vfill \eject

\section{Appendix A: Initial course schedule}

\textbf{This schedule is merely an outline.} For the official calendar, updated and with assignment due dates, go to Blackboard to the \textit{Calendar} link. 

% \textbf{The schedule below is merely an outline, not the official course calendar, .} For  



% The official calendar is linked in the Blackboard sidebar and also available \href{https://calendar.google.com/calendar/embed?src=30ev1uk5php6c8249q1u4noku8%40group.calendar.google.com&ctz=America%2FDetroit}{here}. It is kept up-to-date and is assumed to be correct whenever there is an apparent date/time conflict. 

% Please add the following required packages to your document preamble:
% \usepackage{multirow}
% Please add the following required packages to your document preamble:
% \usepackage{multirow}
% Please add the following required packages to your document preamble:
% \usepackage{multirow}
% Please add the following required packages to your document preamble:
% \usepackage{multirow}
\begin{table}[H]
\begin{tabular}{|c|l|l|}
\hline
Week                & Date    & Topic                                                                \\ \hline
\multirow{3}{*}{1}  & M 8/30  & Welcome + onboarding                                                 \\ \cline{2-3} 
                    & W 9/1   & Integer representation                                               \\ \cline{2-3} 
                    & F 9/3   & Addition and subtraction in binary                                   \\ \hline
\multirow{3}{*}{2}  & M 9/6   & \textit{Labor Day - no class}                                        \\ \cline{2-3} 
                    & W 9/8   & Multiplication and division in binary                                \\ \cline{2-3} 
                    & F 9/10  & The Division Algorithm and MOD operator + Learning Target quiz       \\ \hline
\multirow{3}{*}{3}  & M 9/13  & Propositions and conditional statements                              \\ \cline{2-3} 
                    & W 9/15  & Truth tables and logical equivalence                                 \\ \cline{2-3} 
                    & F 9/17  & Predicates                                                           \\ \hline
\multirow{3}{*}{4}  & M 9/20  & Quantified statements                                                \\ \cline{2-3} 
                    & W 9/22  & Sets and set notation                                                \\ \cline{2-3} 
                    & F 9/24  & \textbf{Learning Target quiz}                                        \\ \hline
\multirow{3}{*}{5}  & M 9/27  & Subsets and set operations                                           \\ \cline{2-3} 
                    & W 9/29  & Functions; domain, codomain, and range                               \\ \cline{2-3} 
                    & F 10/1  & Injective, surjective, and bijective functions; special CS functions \\ \hline
\multirow{3}{*}{6}  & M 10/4  & Additive and Multiplicative Principles of counting                   \\ \cline{2-3} 
                    & W 10/6  & Binomial coefficient, part 1                                         \\ \cline{2-3} 
                    & F 10/8  & \textbf{Learning Target quiz}                                        \\ \hline
\multirow{3}{*}{7}  & M 10/11 & Binomial coefficient part 2                                          \\ \cline{2-3} 
                    & W 10/13 & Open practice with counting strategies                               \\ \cline{2-3} 
                    & F 10/15 & Permutations and $k$-permutations                                    \\ \hline
\multirow{3}{*}{8}  & M 10/18 & Stars-and-bars method                                                \\ \cline{2-3} 
                    & W 10/20 & More open practice with counting strategies                          \\ \cline{2-3} 
                    & F 10/22 & \textbf{Learning Target quiz}                                        \\ \hline
\multirow{3}{*}{9}  & M 10/25 & \textit{Fall Break - no class}                                       \\ \cline{2-3} 
                    & W 10/27 & TBA                                                                  \\ \cline{2-3} 
                    & F 10/29 & \textbf{Learning Target quiz}                                        \\ \hline
\multirow{3}{*}{10} & M 11/1  & Sequences and recursion                                              \\ \cline{2-3} 
                    & W 11/3  & Sigma notation and summing sequences                                 \\ \cline{2-3} 
                    & F 11/5  & Arithmetic and geometric sequences                                   \\ \hline
\multirow{3}{*}{11} & M 11/8  & Determining recurrence relations                                     \\ \cline{2-3} 
                    & W 11/10 & Checking solutions to recurrence relations                           \\ \cline{2-3} 
                    & F 11/12 & \textbf{Learning Target quiz}                                        \\ \hline
\multirow{3}{*}{12} & M 11/15 & Open practice with recurrence relations                              \\ \cline{2-3} 
                    & W 11/17 & Characteristic root method, part 1                                   \\ \cline{2-3} 
                    & F 11/19 & Characteristic root method, part 2                                   \\ \hline
\multirow{3}{*}{13} & M 11/22 & \textbf{Catch-up day and Learning Target Quiz}                       \\ \cline{2-3} 
                    & W 11/24 & \multirow{2}{*}{\textit{Thanksgiving Break - no class}}              \\ \cline{2-2}
                    & F 11/26 &                                                                      \\ \hline
\multirow{3}{*}{14} & M 11/29 & Mathematical induction, part 1                                       \\ \cline{2-3} 
                    & W 12/1  & Mathematical induction, part 2                                       \\ \cline{2-3} 
                    & F 12/3  & Mathematical induction, part 3                                       \\ \hline
\multirow{3}{*}{15} & M 12/6  & \textbf{Learning Target quiz}                                        \\ \cline{2-3} 
                    & W 12/8  & Course retrospective                                                 \\ \cline{2-3} 
                    & F 12/10 & Final exam discussion                                                \\ \hline
\end{tabular}
\end{table}


\vfill \eject

\section{Appendix B: MTH 225 Learning Targets}
\label{sec:learning-targets}

\begin{subsubsection}{Module 1: Computer Arithmetic}
\begin{description}
\tightlist
    \item[CA.1] \textbf{(CORE)} \ I can represent an integer in base 2, 8, 10, and 16 and represent a negative integer in base 2 using two's complement notation.
    \item[CA.2] I can perform addition, subtraction, multiplication, and division in binary. 
\end{description}

\subsubsection{Module 2: Logic}
\begin{description}
\tightlist
    \item[L.1] \textbf{(CORE)} \ I can identify the parts of a conditional statement and write the negation, converse, and contrapositive of a conditional statement.
    \item[L.2] I can construct truth tables for propositions involving two or three variables and use truth tables to determine if two propositions are logically equivalent.
    \item[L.3] I can identify the truth value of a predicate, determine whether a quantified predicate is true or false, and state the negation of a quantified statement. 
\end{description}

\subsubsection{Module 3: Sets and Functions}
\begin{description}
\tightlist
    \item[SF.1] \textbf{(CORE)} \ I can represent a set in roster notation and set-builder notation; determine if an object is an element of a set; and determine set relationships (equality, subset). 
    \item[SF.2] I can perform operations on sets (intersection, union, complement, Cartesian product), determine the cardinality of a set, and write the power set of a finite set.
    \item[SF.3] \textbf{(CORE)} \ I can determine whether or not a given relation is a function; determine the domain, range, and codomain of a function; and find the image and preimage of a point using a function. 
    \item[SF.4] I can determine whether a function is injective, surjective, or bijective.
    \item[SF.5] I can evaluate special computer science functions: floor, ceiling, factorial, \texttt{DIV}, and \texttt{MOD} (\texttt{\%}). 
\end{description}

\subsubsection{Module 4: Combinatorics}
\begin{description}
\tightlist
    \item[C.1] \textbf{(CORE)} \ I can use the additive and multiplicative principles and the Principle of Inclusion and Exclusion to formulate and solve counting problems. 
    \item[C.2] \textbf{(CORE)} \ I can calculate a binomial coefficient and correctly apply the binomial coefficient to formulate and solve counting problems.
    \item[C.3] I can count the number of permutations of a group of objects and the number of $k$-permutations from a set of $n$ objects.
    \item[C.4] I can use the "stars and bars" method to count the number of ways to distribute objects among a group.
\end{description}

\subsubsection{Module 5: Recursion and Induction}
\begin{description}
\tightlist
    \item[RI.1] \textbf{(CORE)} \ I can generate several values in a sequence defined using a closed-form expression or using recursion.
    \item[RI.2] I can use sigma notation to rewrite a sum and determine the sum of an expression given in sigma notation.
    \item[RI.3] I can find closed-form and recursive expressions for arithmetic and geometric sequences.
    \item[RI.4] \textbf{(CORE)} \ I can determine a recurrence relation for a given recursive sequence and check whether a proposed solution to a recurrence relation is valid. 
    \item[RI.5] I can solve a second-order linear homogeneous recurrence relation using the characteristic root method. 
    \item[RI.6] \textbf{(CORE)} \ Given a statement to be proven by mathematical induction, I can state and prove the base case, state the inductive hypothesis, and outline the proof. 
\end{description}


\end{document}
