\documentclass[11pt,letterpaper]{article}

\usepackage{fancyhdr}
\usepackage[latin1]{inputenc}
\usepackage{amsmath}
\usepackage{amsfonts}
\usepackage{amssymb}
\usepackage{graphicx}
\usepackage[hmargin=2cm,vmargin=2.5cm]{geometry}
\usepackage[normalem]{ulem}
\usepackage{enumerate}
\usepackage{hyperref}

\newcommand{\profName}{Prof. Talbert}
\newcommand{\institution}{Grand Valley State University}
\newcommand{\coursename}{MTH 325: Discrete Structures for Computer Science 2}

\pagestyle{fancy}
\setlength\parindent{0in}
\setlength\parskip{0.1in}
\setlength\headheight{15pt}


% \rhead{\workingDate}
\lhead{\textsc{\coursename}}
\rhead{\textsc{\profName}}
\rfoot{\textsc{\thepage}}
% \cfoot{\textit{Built: \today}}
\lfoot{\textsc{\institution}}

% \def\ra{\rightarrow}
% \def\blank{\underline{\hspace{1in}}}
% \def\pageturn{\vfill 
% \begin{flushright}
% 	\begin{small}
% 		Continued $\ra$
% 	\end{small}
% \end{flushright} \newpage}

\begin{document}

\begin{center}
	\begin{Large}
		\textbf{Learning Module: Getting Started} 
		% \textbf{Level ${3:level}}
	\end{Large} \\
	\begin{large}
		Due: Wednesday, January 14, 2015 at 11:59pm 
	\end{large}
\end{center}

\textbf{Description:} Welcome to your first Learning Module. This module will ask you to respond thoughtfully to questions about the course syllabus, the course calendar, the expectations for students and the instructor, the grading system, and how mathematics fits into a degree and career in computing. You'll also be asked to share some information about yourself and to set goals for the semester. 

\smallskip

\fbox{\parbox{6.5in}{
\textbf{SPECIAL OFFER: Students who submit this module before Monday, January 12, 2015 at 11:59pm and receive a Pass mark on it will receive one additional token for the semester.} Remember, each token entitles you to a do-over on a Learning Module, a 24-hour extension, or a makeup of a Timed Module session.}}

% \smallskip

% \textbf{Learning objectives assessed by this module:} 

\smallskip

\textbf{Assessment process:} Please remember that all Learning Modules are graded either \textbf{Pass} or \textbf{No Pass} based on whether your work satisfies the specifications discussed in class. These specifications are housed on the course Blackboard site. \textbf{It is your responsibility to review the specifications and check to make sure your work meets all of them before submitting your work. Please use the checklists provided on Blackboard as a guide.} 

\smallskip

\textbf{Submission:} Please type up your work using \LaTeX \, or a word processor as indicated in the specifications, and submit a PDF of your work as an email attachment to \texttt{mth325gvsu@gmail.com} on or before the due date. Late work will not be accepted for any reason, includng technological reasons. 

\smallskip

\textbf{Academic honesty notice:} While you may discuss your work on this module with other people at the level of general overall ideas, all specific implementations of those ideas and all specific work you do must be your own and not contain anyone else's work. Failure to abide by university academic honesty policies may result in failure not only of the module but of the entire course. 

\section*{Module Activities}

\begin{enumerate}
	\item On Learning Catalytics, you will find an untimed question bank titled \textbf{Syllabus and Calendar Overview} that consists of several questions about information in the syllabus and on the course calendar. \textbf{Answer all of these questions correctly before the deadline for this Learning Module.} You may go back and redo any question that you do not answer correctly, as many times as you want until the deadline. Note that you will need to purchase your Learning Catalytics account in order to be able to finish this module; please see the instructor ASAP if this is an issue for you. 

	\item Carefully read through the document ``Roles and Expectations in MTH 325 for Students and the Instructor''. Then write at least 300 words in response to the following questions: 
		\begin{itemize}
			\item If you are working on Guided Practice and get stuck, what's your plan for getting help before class starts? 
			\item Why is thorough preparation before class so important in MTH 325? 
			\item Why do you think we are using a setup like this in MTH 325, instead of having lecture in class? 
			\item \emph{(Optional, does not count toward the 300-word minimum)} \ Do you have any questions, concerns, or other information you want to share about the course setup and expectations? 
		\end{itemize}

	\item Carefully read through the document ``Grading System for MTH 325''. Then write at least 300 words in response to the following questions: 
		\begin{itemize}
			\item As the document describes, there are no ``points'' in this system. How does a student in MTH 325 keep track of how well he or she is doing in the course? 
			\item As the document describes, there is also no partial credit given on assignments. How does a student in MTH 325 avoid getting marks of ``No Pass'' on Learning Modules and other assignments? 
			\item Why do you think we are using a grading system like this instead of a traditional points-based system? 
			\item \emph{(Optional, does not count toward the 300-word minimum)} \ Do you have any questions, concerns, or other information you want to share about the grading system? 
		\end{itemize}

	\item Read through the article, ``You don't need math skills to be a good developer but you do need them to be a great one'', online at \url{http://bit.ly/1Gt5ac5}. Write at least 500 words that answers the question: \emph{Why should a person interested in computer science take MTH 225 and MTH 325?} Make sure to use some of the points made in the article to answer the question. 

	\item Write at least 500 words to give a ``mathematical biography'' of yourself. Write about your past experiences learning mathematics (not necessarily in school), high and low points in your mathematical background, things you like and dislike about math, and anything else that shaped your mathematical experiences leading up to now. 

	\item Take a moment to think about how you would like to \emph{end} the semester. What kind of accomplishments would you like to have? What kind of knowledge and skills would you like to possess? What kind of experiences would you like to have had? Once you think about this future state for a while, do the following: 
		\begin{itemize}
			\item Set three concrete goals for yourself for this class that are not directly related to the earning of a particular course grade. Each goal must be measurable and realistic. For example, you might set a goal to attend office hours at least once a week; or to make at least three new friends in the class. Then write down each goal along with 1--3 sentences following each goal to describe how you will achieve it. 
			\item Answer the question: \textbf{What grade do you want to earn in this course, and why?} Give at least one good sentence for the ``why'' part. Then, go back to the ``Grading System for MTH 325'' document and make a list of the things you need to do in order to earn that grade. Remember that in our grading system, the grade you earn is entirely up to you -- you pick the grade, and then work toward meeting the requirements for that grade. (Of course, this item is not a contract, and you can change your mind about the grade you want later.) 
		\end{itemize}

	\item (OPTIONAL) \ If you have any questions, concerns, or other information to share, please do so in your writeup and I'll respond if needed. 

\end{enumerate}


\end{document}