\documentclass[11pt,letterpaper]{article}

\usepackage{fancyhdr}
\usepackage[latin1]{inputenc}
\usepackage{amsmath}
\usepackage{amsfonts}
\usepackage{amssymb}
\usepackage{graphicx}
\usepackage[hmargin=2cm,vmargin=2.5cm]{geometry}
\usepackage[normalem]{ulem}
\usepackage{enumerate}
\usepackage{hyperref}

\newcommand{\profName}{Prof. Talbert}
\newcommand{\institution}{Grand Valley State University}
\newcommand{\coursename}{MTH 325: Discrete Structures for Computer Science 2}

\pagestyle{fancy}
\setlength\parindent{0in}
\setlength\parskip{0.1in}
\setlength\headheight{15pt}


% \rhead{\workingDate}
\lhead{\textsc{\coursename}}
\rhead{\textsc{\profName}}
\rfoot{\textsc{\thepage}}
% \cfoot{\textit{Built: \today}}
\lfoot{\textsc{\institution}}

% \def\ra{\rightarrow}
% \def\blank{\underline{\hspace{1in}}}
% \def\pageturn{\vfill 
% \begin{flushright}
% 	\begin{small}
% 		Continued $\ra$
% 	\end{small}
% \end{flushright} \newpage}

\begin{document}

\begin{center}
	\begin{Large}
		\textbf{Learning Module: Getting Started} 
		% \textbf{Level ${3:level}}
	\end{Large} \\
	\begin{large}
		Due: Wednesday, January 14, 2015 at 11:59pm 
	\end{large}
\end{center}

\textbf{Description:} Welcome to your first Learning Module. This module will ask you to respond thoughtfully to questions about the course syllabus, the course calendar, the expectations for students and the instructor, the grading system, and how mathematics fits into a degree and career in computing. You'll also be asked to share some information about yourself and to set goals for the semester. 

\smallskip

\fbox{\parbox{6.75in}{
\textbf{SPECIAL OFFER: Students who submit this module before Monday, January 12, 2015 at 11:59pm and receive a Pass mark on it will receive one additional token for the semester.} Remember, each token entitles you to a do-over on a Learning Module, a 24-hour extension, or a makeup of a Timed Module session.}}

% \smallskip

% \textbf{Learning objectives assessed by this module:} 

\smallskip

\textbf{Assessment process:} Please remember that all Learning Modules are graded either \textbf{Pass} or \textbf{No Pass} based on whether your work satisfies the specifications discussed in class. These specifications are housed on the course Blackboard site. \textbf{It is your responsibility to review the specifications and check to make sure your work meets all of them before submitting your work. Please use the checklists provided on Blackboard as a guide.} 

\smallskip

\textbf{Submission:} Please type up your work using \LaTeX \, or a word processor as indicated in the specifications, and submit a PDF of your work as an email attachment to \texttt{mth325gvsu@gmail.com} on or before the due date. Late work will not be accepted for any reason, includng technological reasons. 

\smallskip

\textbf{Filename designation:} Save your file using the name
\begin{verbatim}
	SectionNo LastName GettingStarted.pdf
\end{verbatim}
where \verb.SectionNo. is your section number (01 or 02) and \verb.LastName. is your last name. For example, if Fred Jackson is in Section 01, his file would be named \verb.01 Jackson GettingStarted.pdf.. Note that there is no space between the word ``Getting'' and the word ``Started''. 

\smallskip

\textbf{Academic honesty notice:} All specific implementations of those ideas and all specific work you do must be your own and not contain anyone else's work. Failure to abide by university academic honesty policies may result in failure not only of the module but of the entire course. 

\section*{Module Activities}

\begin{enumerate}
	\item On Learning Catalytics, you will find a self-paced question bank titled \textbf{Syllabus and Calendar Overview} that consists of several questions about information in the syllabus and on the course calendar. \textbf{Answer all of these questions correctly before the deadline for this Learning Module.} You may go back and redo any question that you do not answer correctly, as many times as you want until the deadline. Note that you will need to purchase your Learning Catalytics account in order to be able to finish this module; please see the instructor ASAP if this is an issue for you. 


	\item Read through the article, ``You don't need math skills to be a good developer but you do need them to be a great one'', online at \url{http://bit.ly/1Gt5ac5}. Write at least 500 words that answers the question: \emph{Why should a person interested in computer science take MTH 225 and MTH 325?} Make sure to use some of the points made in the article to answer the question. 

	\item Write a ``mathematical biography'' of yourself, of at least 500 words in length. Address such issues as: 
		\begin{itemize}
		 	\item What have your past mathematics experiences been like (both bad and good)? 
		 	\item What are some things that you enjoy about mathematics? What are some things that you don't enjoy as much? 
		 	\item What other formative experiences in your mathematical background would you like to share? 
		 \end{itemize} 

	\item The final item in this Learning Module is intended to have you think about and then set concrete goals for yourself this semester. 
		\begin{itemize}
			\item First, go read this document on the idea of SMART goals: 
			\begin{center}
				\url{http://www.hr.virginia.edu/uploads/documents/media/Writing_SMART_Goals.pdf}
			\end{center}
			Once you're done, write a brief summary of this document that includes a description of what the letters in ``SMART'' stand for. 

			\item Next, \textbf{set three goals for yourself in this class that are not directly related to earning a particular grade}. For example, you might set a goal to attend office hours at least once a week; or to learn enough math to enhance a particular software project you have been dreaming about; or to make at least three new friends in the class. For each of those three goals, use the ``S.M.A.R.T. Goal Questionnaire'' in the document linked above to get specific about what the goal will accomplish, how you will measure whether or not the goal has been reached, and so on. Do not turn in the questionnaire form itself, but give your responses in your writeup. There is no minimum word count, but you are expected to give thoughtful effort to this activity. 

			\item Finally, \textbf{set a goal for the grade you wish to earn in the course.} Answer the question: What grade do you want, and why? Then, go to the grade  table in the syllabus and list all of the assignments you need to Pass in the course in order to earn that grade. (You can change your mind about this later; this is not a contract.) Include your intended grade and the list of minimum requirements in the writeup. 

		\end{itemize}
	I highly recommend that you keep a copy of what you type up for this activity and refer back to your goals on a weekly basis, to evaluate how you are doing regarding your progress toward those goals. 

	\item (OPTIONAL) \ If you have any questions, concerns, or other information to share, please do so in your writeup and I'll respond if needed. 

\end{enumerate}


\end{document}